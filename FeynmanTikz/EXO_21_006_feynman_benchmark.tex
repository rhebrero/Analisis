\documentclass[tikz]{standalone}
\usepackage[compat=1.1.0]{tikz-feynman}
% documentation
%https://jpellis.me/projects/tikz-feynman/tikz-feynman/tikz-feynman.pdf

\begin{document}

\begin{tikzpicture}
	\begin{feynman}
	\vertex(a){\({{\rm \Phi}}\)};
	\vertex[right=of a   ] (b);
    \vertex[above right=of b,  yshift =  2mm, xshift = -3 mm] (b1);
	\vertex[below right=of b,  yshift = -2mm, xshift = -3 mm] (b2);
	\vertex[above right=of b1, yshift = -6mm] (f1){\(\bar{\mu}\)};
	\vertex[below right=of b1, yshift =  6mm] (f2){\(\mu\)};
	\vertex[above right=of b2, yshift = -6mm] (f3){\(\bar{f}\)};
	\vertex[below right=of b2, yshift =  6mm] (f4){\(f\)};
    %\vertex[above right=of b2, yshift = -6mm] (f3){};
	%\vertex[below right=of b2, yshift =  6mm] (f4){};
	\diagram* {
		(a)-- [scalar] (b) -- [scalar,edge label'=\({\rm X}\)] (b1),
		(b)-- [scalar,edge label=\({\rm X}\)] (b2),
		(b1)--[anti fermion] (f1),
		(b1)--[fermion] (f2),
		(b2)--[anti fermion] (f3),
		(b2)--[fermion] (f4),
		};
	\end{feynman}
\end{tikzpicture}

\end{document}
